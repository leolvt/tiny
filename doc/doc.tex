\documentclass[12pt]{article}

% ********* Packages Loaded ************
\usepackage{sbc-template}

\usepackage[brazil]{babel}
\usepackage[utf8]{inputenc}
\usepackage{graphicx,url}
\usepackage{subfigure}
\usepackage[portugues,lined,boxed]{algorithm2e}
\usepackage{amsmath}

\sloppy

\title{Linguagem Tiny - Implementação por meio de orientação por objetos \\ Chamada de procedimentos}

\author{Felipe Buzatti Nascimento \and Leonardo Vilela Teixeira}

\address{Departamento de Ciência da Computação \\ Universidade Federal de Minas Gerais (UFMG)
\email{buzatti@dcc.ufmg.br \and vilela@dcc.ufmg.br}
}

\begin{document} 

\maketitle

\begin{resumo} 
Este trabalho expande o interpretador para linguagem tiny, adicionando uma estrutura de procedimentos. Para construir o interpretador foi utilizada uma estrutura de classes seguindo o padrão de projetos interpretador definido em \cite{GoF}.
\end{resumo}


\section{Estrutura de classes}

As novas classes, usadas para implementar a estrutura de procedimentos são: comandoCall, comandoGlobal, comandoLocal, listaProcedimentos, listaVariaveis, Procedimento e Programa.
Foi ainda criada uma classe Erro, relacionada ao tratamento de exceções.

\subsection{classe Programa:} 

Um programa agora é visto como um comando global (facultativo) e uma lista de procedimentos. A classe programa encapsula essa visão.
\begin{algorithm}[h!]
\begin{footnotesize}

	private:\\
		Comando * cmd\_global\;
		ListaProcedimentos * procedimentos\;
	public: \\
		Programa(Comando * g, ListaProcedimentos * p)\;
		~Programa()\;
		void Executa(Contexto\& C)\;

\caption{class Programa}%
\end{footnotesize}
\end{algorithm}

\subsection{classe Procedimento:} 

Procedimentos são formados por um identificador (nome do processo), parametros (facultativos), ComandoLocal (facultativo) e uma lista de comandos.
\begin{algorithm}[h!]
\begin{footnotesize}
  
	private:\\
		std::string * nome\_processo\;
		ListaVariaveis * parametros\;
		Comando * cmd\_local\;
		ListaComandos * lista\_cmds\;
	public:\\
		Procedimento(	std::string * nome\_processo, 
				ListaVariaveis * parametros, 
				Comando * cmd\_local, 
				ListaComandos * lista\_cmds)\;
		~Procedimento()\;
		std::string obtemNome()\;
		ListaVariaveis * obtemParametros()\;
		Comando * obtemLocal()\;
		ListaComandos * obtemComandos()\;
	
\caption{class Procedimento}%
\end{footnotesize}
\end{algorithm}

\subsection{classe Contexto:} 

A classe contexto guarda o estado do programa, bem como sua estrutura. Ela contém os procedimentos do programa (procedimentos\_do\_prog), as variáveis globais (variaveis\_globais) e variáveis locais, armazenadas na pilha de ativação (pilha\_chamada).
\begin{algorithm}[h!]
\begin{footnotesize}
  
	private:\\
		Procedimentos * procedimentos\_do\_prog\;
		Variaveis variaveis\_globais\;
		Pilha pilha\_chamada\;
	public:\\
		Contexto()\;
		~Contexto()\;
		
		double obtemVariavel(char nomeVar)\;
		void defineVariavel(char nomeVar, double valor)\;
		void adicionaVariavel(char nomeVar, double valor)\;
		void adicionaRA()\;
		void removeRA()\;
		Procedimento * obtemProcedimento(std::string nome\_procedimento)\;
		void defineProcedimentos(Procedimentos * procedimentos\_do\_prog)\;
	
\caption{class Contexto}%
\end{footnotesize}
\end{algorithm}

\section{Algoritmos}

Das novas funções, apresentaremos aqui as fundamentais: Programa::Executa, Contexto::AdicionaVariavel, Contexto::AdicionaRA e Contexto::RemoveRA.

\subsection{void Programa::Executa( Contexto\& C )}

Primeira função da interpretação. Adiciona ao contexto a Lista de procedimentos, para futuras chamadas pela função call. Interpreta o comandoGlobal e faz um call do processo main.


\subsection{void Contexto::AdicionaRA()}

Função que faz um "push" da pilha de ativação. Cada vez que um comando call é feito essa função é chamada para empilhar um novo espaço para as variáveis locais.

\subsection{void Contexto::RemoveRA()}

Faz um "pop" da pilha de ativação. Executada ao final de cada chamada de procedimento, para remover o espaço de variáveis usado pelo processo anterior.

\subsection{void Contexto::AdicionaVariavel(char nomevar, double valor)}

Função chamada pelos comandos global e local, para criar, respectivamente, variáveis globais e locais. As variáveis globais são criadas porque a função é chamada com uma pilha de ativação vazia. Quando temos uma chamada de procedimento as variáveis criadas são colocadas na pilha referente a esse procedimento em execução, ouse ja, no topo da pilha.

\section{IMPLEMENTAÇÃO}

Cada classe está implementada em uma dupla de \textit{$<$nome da classe$>$.h} e
\textit{$<$nome da classe$>$.cpp}. Há ainda um arquivo \textit{parser.yy} usado pelo
analisador sintático bison, responsável por produzir a partir desse os arquivos
\textit{parser.h} e \textit{parser.cpp}. Há ainda os arquivos \textit{scanner.ll} e \textit{scanner.h} relativos ao análisador léxico Flex, que gera a partir desses o arquivo \textit{scanner.cpp}.

Encapsulando o parser gerado pela dupla Bison/Flex, há a classe Driver. Essa
classe é responsável por instanciar os analisadores léxico e sintático,
conectá-los, alimentá-los com a entrada (``Código Fonte''), armazenar o
resultado do processo de parsing e tratar erros de parsing encontrados no
processo. A classe fornece os métodos: parse\_stream, parse\_string e 
parse\_file, que iniciam o processo de parsing através de um stream, uma string
ou um arquivo, respectivamente.



\section{Compilação}

O programa deve ser compilado pelo compilador GCC através de um makefile na pasta raiz.

\section{Execução}

O programa deve receber por parâmetro o nome do arquivo-fonte cujo código deve 
ser interpretado.

\subsection{Formato da entrada}

A entrada para o programa é um código fonte na sintaxe da linguagem tiny. A
gramática da linguagem se encontra num arquivo junto ao código fonte do
interpretado. Um exemplo de programa válido seria:

\begin{algorithm}[h!]
\begin{footnotesize}
\begin{verbatim}
proc main()
  local a, b, c;
  read(a);
  read(b);

  while( a > 0 ) do
      c := c + b;
      a := a - 1;
  endw;

  writeVar(c);
  writeln;
endproc

endp
\end{verbatim}	
\caption{Programa válido em tiny}%
\end{footnotesize}
\end{algorithm}



\subsection{Formato da saída}

A saída do interpretador tiny será a saída gerada pelo programa fonte
interpretado, ou mensagens de erro e de requisição de entrada, em caso de erro
ou interpretação de um programa fornecido pela entrada padrão, respectivamente.

\bibliographystyle{sbc}
\bibliography{doc}

\end{document}
