\documentclass[12pt]{article}

% ********* Packages Loaded ************
\usepackage{sbc-template}

\usepackage[brazil]{babel}
\usepackage[utf8]{inputenc}
\usepackage{graphicx,url}
\usepackage{subfigure}
\usepackage[portugues,lined,boxed]{algorithm2e}
\usepackage{amsmath}

\sloppy

\title{Linguagem Tiny - Implementação por meio de orientação por objetos}

\author{Felipe Buzatti Nascimento \and Leonardo Vilela Teixeira}

\address{Departamento de Ciência da Computação \\ Universidade Federal de Minas Gerais (UFMG)
\email{buzatti@dcc.ufmg.br \and vilela@dcc.ufmg.br}
}

\begin{document} 

\maketitle

\begin{resumo} 
Este trabalho produz um interpretador para a linguagem Tiny. Para construir o interpretador foi utilizada uma estrutura de classes seguindo o padrão de projetos interpretador definido em \cite{GoF}.
\end{resumo}


\section{Estrutura de classes}

A estrutura básica de classes consiste em três superclasses virtuais puras, a classe Comando, Expressao e ExpressaoBool, que declaram as três funções utilizadas no processo: Interpreta, Calcula e Avalia.

\subsection{classe Comando:} 

A classe comando define a função Interpreta e é herdada pelas classes ComandoAtribuicao, ComandoFor, ComandoIf, ComandoRead, ComandoWhile, ComandoWrite.
\begin{algorithm}[h!]
\begin{footnotesize}

	public: 
		virtual ~Comando() {}\;
		virtual void Interpreta( Contexto\& C ) = 0\;

\caption{class Comando}%
\end{footnotesize}
\end{algorithm}

\subsection{classe Expressao:} 

Classe relacionada a expressões aritméticas, define a função Calcula. É herdada pelas classes ExpressaoAritmetica e Fator
\begin{algorithm}[h!]
\begin{footnotesize}
  
	public: 
		virtual ~Expressao() {}\;
		virtual double Calcula(Contexto\& C) = 0\;
	
\caption{class Expressao}%
\end{footnotesize}
\end{algorithm}

\subsection{classe ExpressaoBool:} 

Classe relacionada a expressões booleanas, define a função Avalia. É herdada pelas classes ExpressaoBooleana e ExpressaoRelacional.
\begin{algorithm}[h!]
\begin{footnotesize}
  
	public:
		virtual ~ExpressaoBool() {}\;
		virtual bool Avalia (Contexto\& C) = 0\;
	
\caption{class ExpressaoBool}%
\end{footnotesize}
\end{algorithm}

\subsection{classe Contexto:} 

Classe responsável pela estrutura que armazena e fornece acesso às variáveis.
\begin{algorithm}[h!]
\begin{footnotesize}
  
	private:
		std::vector$<$double$>$ variaveis\;

	public:
		Contexto()\;
		~Contexto()\;
		
		double	obtemVariavel(char nomeVar)\;
		void	defineVariavel(char nomeVar, double valor)\;
	
\caption{class Contexto}%
\end{footnotesize}
\end{algorithm}

\subsection{classe ListaComandos:} 

Classe responsável pela estrutura de lista de comandos, que armazena os comandos do programa.
\begin{algorithm}[h!]
\begin{footnotesize}
  
	private:
		std::vector$<$Comando *$>$ lista\_de\_comandos\;

	public:
		ListaComandos(Comando * C)\;
		~ListaComandos()\;
		void AdicionaComando(Comando * C)\;
		void Interpreta(Contexto\& C)\;
	
\caption{class ListaComandos}%
\end{footnotesize}
\end{algorithm}

\section{Algoritmos}

As três funções principais são Interpreta, Avalia e Calcula. Essas são definidas
em cada classe de uma forma diferente, dependendo da classe. A aplicação dessas
funções se dá através de atribuição polimórfica, com essas funções sendo
chamadas para objetos declarados como Comando, ExpressaoBooleana ou Expressao.
Como trata-se de funções virtuais, é chamada a função da classe à qual pertence
o objeto apontado.

\subsection{virtual void Interpreta( Contexto\& C )}

Função típica de comandos. Executa o comando relativo à classe. Por exemplo, na classe ComandoWhile, essa função testa o condicional, e executa a lista de comandos recebida pela classe enquanto a condição for verdadeira. Já na classe ComandoAtribuicao a função recebe o nome da variável e o valor a ser atribuído e através das funções definidas pela classe Contexto modifica o valor da variável.


\subsection{virtual double Calcula(Contexto\& C)}

Função relativa a expressões aritméticas. Faz acesso a variáveis, soma, subtrai, multiplica, ou faz raiz quadrada, dependendo da operação definida pelo objeto. Retorna o resultado do cálculo como um double.

\subsection{virtual bool Avalia (Contexto\& C)}

Função típica de classes que definem expressões booleanas e relacionais. Realiza comparações, AND's, OR's, etc. O resultado dessas operações é um booleano.

\section{IMPLEMENTAÇÃO}

Cada classe está implementada em uma dupla de \textit{$<$nome da classe$>$.h} e
\textit{$<$nome da classe$>$.cpp}. Há ainda um arquivo \textit{parser.yy} usado pelo
analisador sintático bison, responsável por produzir a partir desse os arquivos
\textit{parser.h} e \textit{parser.cpp}. Há ainda os arquivos \textit{scanner.ll} e \textit{scanner.h} relativos ao análisador léxico Flex, que gera a partir desses o arquivo \textit{scanner.cpp}.

Encapsulando o parser gerado pela dupla Bison/Flex, há a classe Driver. Essa
classe é responsável por instanciar os analisadores léxico e sintático,
conectá-los, alimentá-los com a entrada (``Código Fonte''), armazenar o
resultado do processo de parsing e tratar erros de parsing encontrados no
processo. A classe fornece os métodos: parse\_stream, parse\_string e 
parse\_file, que iniciam o processo de parsing através de um stream, uma string
ou um arquivo, respectivamente.



\section{Compilação}

O programa deve ser compilado pelo compilador GCC através de um makefile na pasta raiz.

\section{Execução}

O programa pode receber por parâmetro o nome do arquivo-fonte cujo código deve 
ser interpretado. Se não for informado um arquivo, o interpretador irá
interpretar o código passado pela entrada padrão.

\subsection{Formato da entrada}

A entrada para o programa é um código fonte na sintaxe da linguagem tiny. A
gramática da linguagem se encontra num arquivo junto ao código fonte do
interpretado. Um exemplo de programa válido seria:

\begin{algorithm}[h!]
\begin{footnotesize}
\begin{verbatim}
read(a);
read(b);

while( a > 0 ) do
    c := c + b;
    a := a - 1;
endw;

writeVar(c);
writeln;

endp
\end{verbatim}	
\caption{Programa válido em tiny}%
\end{footnotesize}
\end{algorithm}



\subsection{Formato da saída}

A saída do interpretador tiny será a saída gerada pelo programa fonte
interpretado, ou mensagens de erro e de requisição de entrada, em caso de erro
ou interpretação de um programa fornecido pela entrada padrão, respectivamente.

\bibliographystyle{sbc}
\bibliography{doc}

\end{document}
